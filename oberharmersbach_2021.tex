\documentclass[12pt,aspectratio=169]{beamer}
\usepackage{jmc}
\usepackage{takahashi}
\usepackage{tikz}

\setbeamersize{text margin left = 4em, text margin right = 4em}

\setlength{\parskip}{\baselineskip}

\title{Liquid Tensor Experiment}
\author{Johan Commelin}

\def\Z{\mathbb{Z}}
\def\R{\mathbb{R}}
\def\Ab{\text{Ab}}
\def\TopAb{\text{TopAb}}
\def\AbSh{\text{AbSh}}
\def\ProFin{\text{ProFin}}
\def\CompHaus{\text{CompHaus}}
\def\into{\hookrightarrow}
\def\Cond{\text{Cond}}
\def\Set{\text{Set}}
\def\Mod{\text{Mod}}
\def\ul{\underline}

\begin{document}

\takahashi{\color{title} Liquid Tensor Experiment}

\takahashi{$\Ab$ is nice}

\takahashi{$\TopAb$ is ugly}

\takahashi{\quad $\R^\delta \to \R$ \quad}

\takahashi{$\AbSh(X)$ is nice}

\takahashi{Profinite sets} 

\takahashi{Condensed sets}

\takahashi{$\CompHaus \into \Cond(\Set)$}

\takahashi{$\Cond(\Ab)$ is nice}

\takahashi{Condensed rings/modules}

\takahashi{Six functor formalism}

\takahashi{Analytic rings}

\takahashi{Analytic geometry}

\takahashi{Applications}

\takahashi{Real analysis}

\takahashi{$\{0.d_1d_2d_3\dots \mid d_i = 0,1,\dots,9\}$}

\takahashi{Scholze's challenge}

\takahashi{Some details}

\begin{frame}{Definition}
	An \emph{\color{title} analytic ring} $\mathcal A$
	is a condensed ring $\ul{\mathcal A}$

	\pause

	together with a functor
	\begin{align*}
		\{extr.disc.\} &\to \Mod_{\ul{\mathcal A}}^{\Cond} \\
		S &\mapsto \mathcal A[S]
	\end{align*}

	\pause

	satisfying some conditions.
\end{frame}

\begin{frame}{``Measures''}
	Fix $p \in (0,1] \subset \R$.
	\pause

	For $S$ finite:
	\[
		\R[S]_{\ell^p \le c} =
		\biggl\{ (a_s)_s \,\biggm\vert\, \sum_{s \in S} \|a_s\|^p \le c \biggr\}
	\]
	\pause
	For $S = \varprojlim_i S_i$ profinite:
	\[
		\mathcal M_{p}(S) = \bigcup_c \varprojlim_i \R[S_i]_{\ell^p \le c}
	\]
	\pause
	\[
		\mathcal M_{< p}(S) = \varinjlim_{p' < p} \mathcal M_{p'}(S)
	\]
\end{frame}

\begin{frame}{Main theorem}
	Fix $0 < p \le 1$.
	
	Then $(\R, \mathcal M_{< p})$ is an analytic ring.
\end{frame}

\begin{frame}
	\begin{tikzpicture}
		\draw[color=title,very thick] (-1,0) -- (0,0) -- (0,-2.3) -- (-1,-2.3);
		\node at (-.5,-3) {Theorem 9.5};
		\draw[color=title,very thick] (-1,-3.7) -- (0,-3.7) -- (0,-6) -- (-1,-6);
		\node at (-.5,-6.5) {Main theorem};
		\node at (4,-0.6) {Functional analysis};
		\node at (4,-1.4) {Homological algebra};
		\node at (4,-4.6) {Condensed mathematics};
		\node at (4,-5.4) {Categorical reduction steps};
	\end{tikzpicture}
\end{frame}

\begin{frame}
	$\Z((T)) \to \R$

	\pause

	Polyhedral lattices

	\pause

	Normed exactness

	\pause

	Chase inequalities through spectral sequences
\end{frame}

\begin{frame}{Joint work with}
	\begin{itemize}
		\item Peter Scholze
	\end{itemize}

	The Lean community
	\begin{itemize}
		\item Damiano Testa
		\item Patrick Massot
		\item Kevin Buzzard
		\item Riccardo Brasca
		\item Adam Topaz
		\item Scott Morrison
	\end{itemize}
\end{frame}

\end{document}
